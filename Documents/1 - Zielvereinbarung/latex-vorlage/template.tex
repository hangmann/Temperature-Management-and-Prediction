%
% Copyright (C) 2001 by Holger Karl,
% karl@ft.ee.tu-berlin.de
%
% file: template.tex
%
% Time-stamp: Sat Oct 06 08:29:52 2001
%
% Template fuer die Ausarbeitungen zu TKN-Seminaren
%
\documentclass[12pt,twoside,doublepage]{article}


% Hier den Namen des Teilnehmers und den Titel  der Ausarbeitung eintragen:
\newcommand{\teilnehmer}{Hendrik Hangmann}
\newcommand{\ausarbeitung}{Generating Adjustable Temperature Gradients on modern FPGAs}


% Falls die Ausarbeitung in Deutsch erfolgt,
% die folgenden Kommentar-Zeichen '%' entfernen, andernfalls diese
 % Kommandos auskommentiert lassen:
% Languages:
% \usepackage[german]{babel}
% \usepackage[T1]{fontenc}
% \usepackage[latin1]{inputenc}
% \selectlanguage{german}

%%%%%%%%%%%%%%%%%%%%%%%%%%%
% Im restlichen Vorspann KEINE Aenderungen machen!
%%%%%%%%%%%%%%%%%%%%%%%%%%%
\usepackage{times}
\usepackage{url}

\usepackage{geometry}

%\usepackage{psfraq}

\geometry{a4paper,body={5.8in,9in}}

% Graphics:
\usepackage{graphicx}
% aller Bilder werden im Unterverzeichnis figures gesucht:
\graphicspath{{figures/}}

% Headers:
\usepackage{fancyhdr}
% \pagestyle{fancy}
\pagestyle{fancy}
\fancyhead{}
%\fancyhead[LE]{ \teilnehmer}
\fancyhead[L]{ \teilnehmer}
\fancyhead[LO]{}
\fancyhead[RE]{}
%\fancyhead[RO]{ \ausarbeitung}
\fancyhead[R]{ Target agreement}
\fancyfoot[L]{ \today }
\fancyfoot[C]{  }
\fancyfoot[R]{ \thepage }
\renewcommand{\footrulewidth}{0.5pt}

\begin{document}

\title{\ausarbeitung}
\author{\teilnehmer}
\maketitle
\thispagestyle{empty}

%%%%%%%%%%%%%%%%%%%%%%%%%%%%%%%%%%%%%%
% ab hier steht der eigentliche Text:


% Abstract gives a brief summary of the main points of a paper:
%\begin{abstract}
%This paper gives an overview of TCP Westwood (TCPW), its predecessor TCP Reno and its successors TCP Westwood+ (TCPW+) and TCP LogWestwood+.
%  TCPW is a Transmission Control Protocol, which is based on the widely-used TCP Reno. Additionally TCPW uses an end-to-end bandwidth estimation to adaptively set the congestion window and slowstart treshold after an congestion episode. This enhancement clearly outperforms TCP Reno's strategy of simply halving the congestion window.
%  However, TCPW's bandwidth estimation algorithm can't handle ACK compression and thus, fails to work in the real internet. TCPW+ enhances this algorithm and works as desired. \\ All of these protocols work with a linear increase of the congestion window during the congestion avoidance phase, but TCP LogWestwood+. TCP LogWestwood+ combines the bandwidth estimation algorithm of TCPW+ with an logaritmic increase of the congestion window, which makes it even more aggressive.
%\end{abstract}

% the actual content, usually separated over a number of sections
% each section is assigned a label, in order to be able to put a
% crossreference to it

\section{Motivation}
\label{sec:motivation}

\begin{itemize}
	\item hitze ist schlecht, vor allem bei kleiner werdenden schaltungen und chips 
	\item Heat generation never been studied practicly
	\item ermoeglicht forschungen im bereich ausgleichung von thermischen ungleichheiten und hot spots
	\item hot spots erzeugen, heat generating circuits entwickeln 
\end{itemize}
 

\section{Formulation of problem}
\label{sec:problem}

\begin{itemize}
	\item m�glichst viel Hitzer erzeugen
	\item heat generating cores, with FlipFlops, LUTs, DSPs, DRAMS
	\item mix aus den Komponenten
	\item taktung
	\item 
\end{itemize}

\section{Objective target}
\label{sec:target}

Ziel ist es

\section{Time management}
\label{sec:time}



This document has already introduced the most important constructs of
\LaTeX. What is necessary to produce documents with \LaTeX is simple
any normal text editor and a \LaTeX distribution. This is commonly
installed on practically all UNIX-type systems; for Windows, an
excellent \LaTeX exists, called MikTeX, available from
\url{www.miktex.org}. Almost all distributions come with a large patch
of examples and introductory material; consult your local installation
for details. 

Lots of supplementary and background information, FAQs, etc.\ is
available from the Comprehensive TeX Archive Network (CTAN); the
German mirror of which is \url{www.dante.de}. 

%%%%%%%%%%%%%%%%%%%%%%%%%%%%%%%%%%%%%%
% hier werden - zum Ende des Textes - die bibliographischen Referenzen
% eingebunden
%
% Insbesondere stehen die eigentlichen Informationen in der Datei
% ``bib.bib''
%
\bibliography{bib}
\bibliographystyle{plain}

\end{document}


