%!TEX root = /Users/nphs/Dropbox/0-Bachelor/2012-BA-Cioran/02-Arbeit/BA-Cioran.tex
\chapter{Konzeption und Modellierung}
\label{ch:konzeption}
Dieses Kapitel gibt einen �berblick �ber die zu implementierenden Funktionen und das \ac{UI} der zu entwickelnden Applikation \emph{ginkgo mobile}. Das Ziel dabei ist eine neue Erfahrung f�r die Teilnehmer von wissenschaftlichen Konferenzen zu erm�glichen. Daf�r werden im Abschnitt \ref{sec:ginkgoFeatures} die f�r \emph{ginkgo mobile} relevanten Funktionen des \ac{CMSs} \emph{ginkgo} erkl�rt. Die wichtigsten daraus resultierenden \emph{Use Cases} und die notwendigen Komponenten zur Implementierung werden im Abschnitt \ref{sec:ginkgomobile} beschrieben. 

\section{ginkgo Features}
\label{sec:ginkgoFeatures}
Registrierte Benutzer haben ein eigenes Profil, indem Informationen �ber die eigene Person, wie Interessen und Arbeitgeber, angezeigt werden. Durch die Profile ist jeder Nutzer identifizierbar. Dadurch k�nnen Nutzer anderen Nutzern \emph{folgen},  sich private Nachrichten senden und Status Updates posten. \graffito{Activity Stream} �ber die Liste der registrierten Veranstaltungen lassen sich Informationen zu einzelnen Events anzeigen, die im \emph{veranstaltungsbezogenen Activity Stream} auf der jeweiligen Seite der Konferenz angezeigt werden. Dabei handelt es sich beispielsweise um ge�nderte Fristen, angenommene Paper und neue Reviewer. Des Weiteren lassen sich Personen anzeigen, die dem jeweiligen Event \emph{folgen} oder an der Veranstaltung \emph{teilnehmen} wollen. Durch die M�glichkeit einem Event zu \emph{folgen}, werden die Informationen des jeweiligen \emph{veranstaltungsbezogenen Activity Stream} auch in dem pers�nlichen \emph{Global Activity Stream} angezeigt. Dieser zeigt zus�tzliche Informationen zu Status Updates von dem jeweiligen Nutzer und seinen Freunden an. 

\section{ginkgo mobile}
\label{sec:ginkgomobile}
\subsection{Use Cases}
Im folgenden Abschnitt wird eine Auswahl an wichtigen \emph{Use Cases} gegeben.
\subsubsection{Informationen zu Usern und Events}
Der Nutzer soll Informationen zu jedem \emph{User} und zu jedem \emph{Event} abfragen k�nnen.
\subsubsection{Die Anzeige von Activity Streams}
Zu jedem \emph{Event} und jedem \emph{User} soll der jeweilige \emph{Activity Stream} angezeigt werden k�nnen. Dar�ber hinaus soll dem Nutzer die M�glichkeit zum verfassen von eigenen Status Updates gegeben werden. 

\subsubsection{Private Nachrichten}
Kommunikation zwischen Nutzern soll durch private Nachrichten erm�glicht werden.

\subsubsection{Landkarte}
Auf einer Landkarte sollen dem Nutzer alle umliegende \emph{Events} angezeigt werden.

\subsubsection{Friends/Follower anzeigen}
Durch die \emph{Friends} und \emph{Follower} eines \emph{Users} soll durch einen \emph{ViewPager} navigiert werden k�nnen.

\subsection{Geplante Android Komponenten und deren Verhalten}
\label{sec:ginkgomobilekomponenten}
Die neu zu erstellende \emph{LoginActivity} wird als Einstiegspunkt in \emph{ginkgo mobile} dienen. Nach Eingabe der Benutzerdaten wird bei einer erfolgreichen Authentifizierung die \emph{DashActivity} gestartet (siehe Abbildung \ref{pic:mock_globaldash}). Diese besteht aus dem DashboardFragment und einer View (siehe Abschnitt \ref{sec:androidprog}), die als Container f�r Fragments dient. Zu Beginn enth�lt der Container das \emph{ActivitystreamFragment} um den \emph{Global Activity Stream} (siehe Abschnitt \ref{sec:ginkgoFeatures}) des eingeloggten Nutzers anzuzeigen.
\begin{figure}[h]
	\includegraphics[width=\textwidth]{__pics/diagramme/dashkonzept_blockd}
	\caption{Nach einem erfolgreichen Login wird die DashActivity gestartet. Diese besteht aus dem DashboardFragment und einem Container der andere Fragments beinhaltet.}
	\label{globalDash}
\end{figure}

Durch Interaktionen des Nutzers mit dem \emph{DashboardFragment} kann die \emph{DashActivity} das zur Zeit im Container angezeigte Fragment durch ein anderes ersetzen. Die m�glichen Fragments, die der Container beinhalten kann, werden im Folgenden erkl�rt. 
\begin{itemize}
	\item \textbf{UserFragment}
	
	Das {UserFragment} kann sowohl das Profil des eingeloggten Nutzers als auch Profile anderer Personen anzeigen. Dazu geh�ren pers�nliche Informationen und der jeweilige \emph{Global Activity Stream}. Um den jeweiligen Nutzer zu folgen (siehe Abschnitt \ref{sec:ginkgoFeatures}) gen�gt ein Klick auf den \emph{Follow} Button. �ber den \emph{Message} Button kann der Person eine private Nachricht schreiben (siehe Abbildung \ref{pic:mock_profile}).
	\item \textbf{ActivitystreamFragment}
	
	Durch das \emph{ActivitystreamFragment} k�nnen sowohl \emph{Global Activity Streams} als auch \emph{eventbezogene Activity Streams} angezeigt werden. Um die Funktionen dieses Fragments wiederzuverwenden, wird es zus�tzlich in dem \emph{UserFragment} und \emph{EventFragment} mit eingebunden.
	 
	\item \textbf{CheckinFragment}
	
	Dem Nutzer werden auf einer Weltkarte alle ihm naheliegenden Veranstaltungen angezeigt (siehe Abbildung \ref{pic:mock_checkin}).

	\item \textbf{ConversationFragment}
	
	Es werden alle Konversationspartner des eingeloggten Nutzers angezeigt. Durch einen Klick auf eine Person wird der bereits mit ihm gef�hrte Dialog angezeigt.

	\item \textbf{EventlistFragment}
	
Es werden alle auf \emph{ginkgo} registrierte Veranstaltungen mit ihren Rahmeninformationen aufgelistet (siehe Abbildung \ref{pic:mock_eventlist}). Zus�tzlich wird dem Nutzer die M�glichkeit geboten ein neues Event zu erstellen.

	\item \textbf{EventFragment}
	
	Durch die Auswahl einer Veranstaltung im \emph{EventlistFragment} wird das \emph{EventFragment} mit detaillierten Informationen zu dem jeweiligen Event angezeigt. Der Nutzer kann sich informieren welche Personen diesem Event folgen oder an der Veranstaltung teilnehmen wollen. �ber die Schedule Schaltfl�che soll einen �berblick �ber die Sessions angezeigt werden (siehe Abbildung \ref{pic:mock_event}). 
	\item \textbf{FollowersFragment}
	
	Im \emph{FollowersFragment} werden die Freunde und Follower eines Nutzers angezeigt. Zwischen diesen Kategorien kann mittels \emph{ViewPager} (siehe Abschnitt \ref{sec:supportpackage}) umgeschaltet werden.
\end{itemize} 

\subsection{Actionbar}
\label{sec:actionbar}
Durch die Modularisierung der einzelnen Fragments k�nnen Funktionen die global vorhanden sein sollen, in der Actionbar der \emph{DashActivity} implementiert werden (siehe \ref{sec:android3}). Dazu geh�ren die M�glichkeiten ein Status Update abzusetzen und ein Logout aus der Applikation durchzuf�hren. Dabei wird das gespeicherte Access Token gel�scht. Dadurch wird bei dem n�chsten Start der Applikation erneut der Benutzername und das Passwort abgefragt, sodass sich eine andere Person einloggen kann.

\subsection{\ac{UI} Skizzen}

	\begin{figure}[h]
		\includegraphics[width=\textwidth]{__pics/mock/globalDash}
		\caption{Die \emph{DashActivity} wird angezeigt, nachdem der User sich mit seinem Benutzernamen und Passwort eingeloggt hat.}
		\label{pic:mock_globaldash}
	\end{figure}
	\begin{figure}
		\includegraphics[width=\textwidth]{__pics/mock/eventList}
		\caption{Ein �berblick �ber die bei \emph{ginkgo} registrierten Veranstaltungen wird im \emph{EventlistFragment} angezeigt.}
		\label{pic:mock_eventlist}
	\end{figure}
	\begin{figure}
		\includegraphics[width=\textwidth]{__pics/mock/oneEvent}
		\caption{Das \emph{EventFragment} zeigt Informationen zu einem speziellen Event an.}
		\label{pic:mock_event}
	\end{figure}
	\begin{figure}
		\includegraphics[width=\textwidth]{__pics/mock/strangerProfile}
		\caption{Das \emph{UserFragment} zeigt die pers�nlichen Informationen eines Benutzers an.}
		\label{pic:mock_profile}
	\end{figure}

	\begin{figure}
		\includegraphics[width=\textwidth]{__pics/mock/checkIn}
		\caption{Es k�nnen alle in der unmittelbaren Umgebung des Benutzers stattfindende Veranstaltungen angezeigt werden.}
		\label{pic:mock_checkin}
	\end{figure}

	\begin{figure}
		\includegraphics[width=\textwidth]{__pics/mock/profilePop}
		\caption{Durch den Klick auf das eigene Profilbild hat der Nutzer die M�glichkeit ein Status Update zu setzen und sich auszuloggen.}
		\label{pic:mock_status}
	\end{figure}
	
	\begin{figure}
		\includegraphics[width=\textwidth]{__pics/mock/publications}
		\caption{Es lassen sich zu jeder Person alle Publikationen anzeigen.}
		\label{pic:mock_publications}
	\end{figure}

%%\begin{figure}
%%			\includegraphics[width=\textwidth]{globalDash}
%%			\caption{Die Dash Activity wird angezeigt, nachdem der User sich mit seinem Benutzernamen und Passwort eingeloggt hat}
%%			\label{globalDash}
%%			\end{figure}
%%			\begin{figure}
%%			\includegraphics[width=\textwidth]{eventList}
%%			\caption{Die EventList Activity zeigt alle Events an}
%%			\label{eventList}
%%			\end{figure}
%%			\begin{figure}
%%			\includegraphics[width=\textwidth]{oneEvent}
%%			\caption{Die ShowEvent Activity zeigt die Informationen zu einem speziellen Event}
%%			\label{oneEvent}
%%			\end{figure}
%%			\begin{figure}
%%			\includegraphics[width=\textwidth]{strangerProfile}
%%			\caption{showProfile Activity zeigt die pers�nlichen Informationen eines Benutzers an}
%%			\label{profile}
%%			\end{figure}
%%			\begin{figure}
%%			\includegraphics[width=\textwidth]{checkIn}
%%			\caption{checkIn Activity zeigt die naheliegenden Events an}
%%			\label{checkIn}
%%		\end{figure}
