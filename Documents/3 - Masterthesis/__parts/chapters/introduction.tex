 
\chapter{Introduction}
\label{ch:intro}
\section{Motivation}
Nowadays there is a large field of research in thermal effects on \ac{VLSI} systems. Due to their shrinking device structures and the ever increasing power consumption, these state-of-the-art devices may cause dramatically increased thermal effects and high temperatures. This again could have consequences on the function and reliability of \ac{VLSI} circuits. By the reason that the carrier mobility is degraded and the interconnect resistivity increases, the devices could suffer from decreased execution speed and longer interconnect delays \cite{Huang2006}, which is out of place in hard real-time systems and other today's fields of applications. Beyond that those timing errors can eventually lead to the premature occurrence of hard errors \cite{Borkar2005}. Hence local hot spots and high temperature gradients can lead to a shortened device life time and a shrunk package reliability.

As a start it is highly recommended to monitor the on-chip temperature. For that reason many devices feature a built-in temperature sensor. But moreover it is important to predict the temperature at any place on the die at a given heat flow. By knowing the thermal conductivity of the die and its layers there are several ways to model the heat flow and to predict the die's temperature. 
In cases that the die's structure, e.\,g. the number of layers and its thermal conductivity is not known, for example in early-state \ac{VLSI} design or \ac{FPGA} based systems, reconfigurable devices may only require the on-line learning of a heat flow and temperature model.

In order to achieve this, \acp{FPGA} benefit from their reconfigurability. Thus it is possible to implement heat generating circuits on the one hand and a network of temperature sensors on the other hand. In combination with a \ac{RC} network, which is commonly used for heat flow prediction models, it is possible to generate and learn a temperature model suitable for \acp{FPGA}. 

Unlike the heat models that use physical device information, which derive the \ac{RC} network's parameters by the die's physical characteristics,
\ac{FPGA} based heat models need to learn this parameters on-line. For means of calibration and parameter optimization it is mandatory to heat up the die, both uniformly and by creating hot spots, i.\,e. high temperature spots on the die.
Other approaches which are using the on-line learning model have the drawback that it may be infeasible to start with a fully random initial parameter set, as they start learning with an almost optimal solution. Furthermore, randomized hill climbing is used for the purpose of optimizing, which may also be improvable for fully random initial parameters \cite{Happe}, e.\,g. by using algorithms like Simulated Annealing or Paricle Swarm Optimization. Furthermore, improved heat generating circuits \cite{Agne2013} and temperature sensors \cite{Ruthing2012} may also enhance the on-line heat model.

\section{Thesis Structure}
Section~\ref{ch:heatmodeling} gives an overview of heat models, that are derivable from the die's physical characteristics. Also the \ac{RC} networks will be introduced in Section~\ref{ch:rcn}. Building on those, Section~\ref{ch:ext} explains the basic extension to an online heat model. In the following the needed temperature sensors and heat generating circuits are described. 
Section~\ref{ch:onlineevo} describes the online heat model in detail. In the following several optimization algorithms are introduced, whose results are presented and evaluated in Section~\ref{ch:experiments}. 
Section~\ref{ch:conclusion} summarizes and concludes this thesis and provides an outlook.