%!TEX root = /Users/nphs/Dropbox/0-Bachelor/2012-BA-Cioran/02-Arbeit/BA-Cioran.tex
\chapter{Problemdefinition}
\label{ch:problem}

There are several different ways to develop dedicated heat- generating circuits on a \ac{FPGA}. The \ac{FPGA} consists of three main parts: Slices, \acp{BRAM} and \acp{DSP}. These components get wired dynamically, depending on the desired functionality. The functionality is described in Hardware Description Languages such as \acp{VHDL}. Slices consists of several \acp{LUT} and \acp{FF} and can be combined to any desired circuit. \acp{DSP} are hard-\,coded circuits for special uses, e.\,g. calculations. By the reason that e.\,g. multipliers would take a high amount of space if they were realized by \acp{LUT} and \acp{FF}, these tasks are mapped to special processors for that purpose. Because of their special purpose, they are highly space-\,saving. \acp{BRAM} are memory units. It is possible to instantiate Slices (\acp{LUT} and \acp{FF}), \acp{DSP} or \acp{BRAM} and use them specifically for heat-\,generating circuits. During the research it is eligible to identify the part that generates the most heat and research if the mixture of parts outperforms that solution. Furthermore the frequencies of these parts are varied to measure the influence of diverse clocking.